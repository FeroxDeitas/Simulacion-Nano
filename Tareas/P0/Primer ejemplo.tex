\documentclass{article}
\setlength{\parskip}{5pt} % esp. entre parrafos
\setlength{\parindent}{0pt} % esp. al inicio de un parrafo
\usepackage{amsmath} % mates
\usepackage[sort&compress,numbers]{natbib} % referencias
\usepackage{url} % que las URLs se vean lindos
\usepackage[top=25mm,left=20mm,right=20mm,bottom=25mm]{geometry} % margenes
\usepackage{hyperref} % ligas de URLs
\usepackage{graphicx} % poner figuras
\usepackage[spanish]{babel} % otros idiomas
\usepackage{textcomp}
\usepackage{pgfplots} % crear graficas
\pgfplotsset{width=10cm,compat=1.9}

\title{Tarea P0} % titulo
\author{Jorge Torres} % author
\date{\today}

\begin{document} % inicia contenido

\maketitle % cabecera

\begin{abstract} % resumen
  Es simplemente una demo sencilla del uso b\'{a}sico de \LaTeX{} en
  Overleaf.
\end{abstract}

\section{Introducci\'{o}n}\label{intro} % seccion y etiqueta



Este es un texto ejemplo para que hagan los reportes de sus
tareas. Vamos a incluir una ecuaci\'{o}n \eqref{equ}:
\begin{equation}
  f(x) = 2 \sin(x) - \int_0^\infty \frac{1}{1 + x} \text{d}x.
  \label{equ}
\end{equation}

\begin{figure} % figura
    \centering
    \includegraphics[width=60mm]{limon.jpg} % archivo
    \caption{Lim\'{o}n tomado de \url{https://www.elmundo.es/elmundo/2011/01/25/ciencia/1295977576.html} con licencia CC.}
    \label{limon}
\end{figure}

\newpage

Vamos a aprender adem\'{a}s a citar fuentes \citep{ejemplo}. Incluimos un
cuadro \ref{datos} con algunos datos y en la figura \ref{limon} hay un
lim\'{o}n.

\begin{table} % cuadro
    \caption{Ocupo explicar de qu\'{e} se trata mi cuadro.} % explicacion
    \label{datos} % etiqueta
    \centering % centrar
    \begin{tabular}{l|cr} % izq sep centrada der
         Algo & $\beta$ & 10.220 \\
         Otro & $\alpha$ & 1932.323
    \end{tabular}
\end{table}

\begin{table}
    \caption{Aqu\'{i} otro ejemplo de tabla.}
    \label{otros_datos}
    \centering
    \begin{tabular}{c|c|c|c}
      Info 1 & Info 2 & $\Omega$ & Info 3 \\
      Dato 1 & Dato 2 & $\pi$ & Dato 3
    \end{tabular}
\end{table}

\section{Gr\'{a}ficas}
En esta secci\'{o}n se aprende c\'{o}mo crear gr\'{a}ficas en LaTeX

\begin{center}
\begin{tikzpicture}
\begin{axis}[
title={Dependencia de la solubilidad de CuSO\(_4\cdot\)5H\(_2\)O con la Temperatura},
xlabel={Temperatura [\textcelsius]},
ylabel={Solubilidad [g por 100 g de agua]},
xmin=0, xmax=100,
ymin=0, ymax=120,
xtick={0,20,40,60,80,100},
ytick={0,20,40,60,80,100,120},
legend pos=north west,
ymajorgrids=true,
grid style=dashed,
]
\addplot[
color=blue,
mark=square,
]
coordinates{
(0,23.1)(10,27.5)(20,32)(30,37.8)(40,44.6)(60,61.8)(80,83.8)(100,114)
};
\legend{CuSO\(_4\cdot\)5H\(_2\)O}

\end{axis}
\end{tikzpicture}
\end{center}

\section{Conclusiones}\label{conclu}

En este documento no m\'{a}s se hizo una intro en la secci\'{o}n \ref{intro}.

\bibliography{simu}
\bibliographystyle{plainnat}

\end{document}
