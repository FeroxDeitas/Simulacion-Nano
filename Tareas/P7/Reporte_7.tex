\documentclass{article}
\setlength{\parskip}{0pt} % esp. entre parrafos
\setlength{\parindent}{20pt} % esp. al inicio de un parrafo
\usepackage{amsmath} % mates
\usepackage{listings}
\usepackage{xcolor}
\usepackage[sort&compress,numbers]{natbib} % referencias
\usepackage{url} % que las URLs se vean lindas
\usepackage[top=10mm,left=20mm,right=20mm,bottom=25mm]{geometry} % \textbf{\textbf{}}margenes
\usepackage{hyperref} % ligas de URLs
\usepackage{graphicx} % poner figuras
\usepackage{caption}
\usepackage{subcaption}
\usepackage[spanish]{babel} % otros idiomas
\hypersetup{
    colorlinks=true,
    linkcolor=blue,
    filecolor=blue,      
    urlcolor=blue,
}
\renewcommand{\lstlistingname}{C\'odigo}
\definecolor{codegreen}{rgb}{0,0.6,0}
\definecolor{codegray}{rgb}{0.5,0.5,0.5}
\definecolor{codepurple}{rgb}{0.58,0,0.82}
\definecolor{backcolour}{rgb}{0.95,0.95,0.92}
\lstdefinestyle{mystyle}{
    backgroundcolor=\color{backcolour},   
    commentstyle=\color{codegreen},
    keywordstyle=\color{magenta},
    numberstyle=\tiny\color{codegray},
    stringstyle=\color{codepurple},
    basicstyle=\ttfamily\footnotesize,
    breakatwhitespace=false,         
    breaklines=true,                 
    keepspaces=true,                 
    numbers=left,                    
    numbersep=5pt,                  
    showspaces=false,                
    showstringspaces=false,
    showtabs=false,                  
    tabsize=2
}
\lstset{style=mystyle}

\title{Reporte 7:\\B\'usqueda Local}
\author{Jorge Torres}
\date{\today}

\begin{document}

\maketitle

\section{Objetivo}\label{obj}
La actividad consiste en maximizar una variante de una funci\'on bidimensional, $g(x, y)$, por medio de una b\'usqueda local aleatoria con restricciones en los ejes. La posición actual es un par $x, y$ y se realizan dos movimientos aleatorios, $\Delta x$ y $\Delta y$, cuyas combinaciones posibles proveen ocho posiciones vecino, de los cuales aquella que logra el mayor valor para $g$ es seleccionado. El resultado final es un compendio de los valores m\'aximos encontrados para 10, 20 y 30 r\'eplicas simult\'aneas de 100, 1000 y 10,000 pasos de la b\'usqueda. Adem\'as, en la secci\'on \ref{res} se visualizan algunos de los pasos que se realizaron para la b\'usqueda de 10 r\'eplicas en una proyecci\'on plana de la funci\'on.

\section{Desarrollo}\label{des}
El desarrollo de la pr\'actica est\'a basado en el \href{https://github.com/satuelisa/Simulation/blob/master/LocalSearch/minimize1D.py}{c\'odigo} implementado por E. Schaeffer \cite{elisa1}, con el cual realiza una b\'usqueda de valores m\'inimos para una funci\'on de una sola variable. En \'esta actividad, se selecciona la ecuaci\'on \ref{eq1} como funci\'on a evaluar por la particular cantidad de valores m\'aximos y m\'inimos locales que presenta, lo que podr\'ia representar una dificultad para encontrar un m\'aximo total. La ecuacu\'on se define en el c\'odigo \ref{codigo1}.

\begin{equation}\label{eq1}
    g(x, y) = \cos^2{x} + 0.3x + \sin^2{y} - 0.6y
\end{equation}

\begin{lstlisting}[caption=Funci\'on a Evaluar, label=codigo1, language=Python]
def g(x, y):
    px = (cos(x))**2 + 0.3 * x
    py = (sin(y))**2 - 0.6 * y
    return px + py
\end{lstlisting}

En el c\'odigo \ref{codigo2} se definen algunos par\'ametros con los que se eval\'ua la funci\'on, como el rango de los ejes, la resoluci\'on con que se le grafica y el valor m\'aximo que pueden tomar los diferenciales de movimiento, $\Delta x$ y $\Delta y$.

\begin{lstlisting}[caption=Par\'ametros de Evaluaci\'on, label=codigo2, language=Python]
low = -10
high = -low
step = 0.05
p = np.arange(low, high, step)
n = len(p)
z = np.zeros((n, n), dtype=float)
valores=[]
paso = 0.5
\end{lstlisting}

Para graficar la funci\'on en una proyecci\'on plana, se necesita evaluar la funci\'on en una cantidad finita de puntos, dada por la intrucci\'on \texttt{p}, dentro del rango previamente definido, como se observa en el c\'odigo \ref{codigo3}.

\begin{lstlisting}[caption= Puntos para Graficar la Funci\'on, label=codigo3, language=Python]
for i in range(n):
    x = p[i]
    for j in range(n): 
        y = p[n - j - 1] # voltear
        z[i, j] = g(x, y)
        valores.append(g(x, y))
\end{lstlisting}

Por medio del c\'odigo \ref{codigo4} se definen las coordenadas de los agentes o puntos que realizar\'an la b\'usqueda. Adem\'as, tambi\'en se definen las coordenadas iniciales para el mejor valor entre los agentes creados.

\begin{lstlisting}[caption= Creaci\'on de Coordenadas para B\'usqueda, label=codigo4, language=Python]
tmax=10
for a in range(10, 31, 10):

    resultados = pd.DataFrame()
    for tiem in range(2, 5):
        agentes =  pd.DataFrame()
        agentes['x'] = [uniform(low, high) for i in range(a)]
        agentes['y'] = [uniform(low, high) for i in range(a)]
        agentes['best'] = [min(valores) for i in range(a)]
        bestx = agentes['x'][0]
        besty = agentes['y'][0]
        best = g(bestx, besty)
\end{lstlisting}

En el c\'odigo \ref{codigo5}, utilizando los valores de \texttt{tmax} y \texttt{tiem} definidos anteriormente, se consigue la cantidad de pasos que dar\'an los agentes. Los diferenciales de paso se deciden de manera aleatoria y se agregan a la lista.

\begin{lstlisting}[caption= Diferenciales de Paso, label=codigo5, language=Python]
        for tiempo in range(tmax**tiem):
            agentes['dx'] = [uniform(0, paso) for i in range(a)]
            agentes['dy'] = [uniform(0, paso) for i in range(a)]
\end{lstlisting}

Para que cada agente realice movimientos en cada una de las ocho posibles direcciones, se implementa el c\'odigo \ref{codigo6}. Adem\'as, se eval\'ua la funci\'on en las ocho direcciones para posteriormente poder determinar cu\'al presenta un valor mayor. Las l\'ineas \texttt{9} a \texttt{16} impiden que cualquiera de los agentes realice un movimiento que se encuentre fuera del rango evaluado.

\begin{lstlisting}[caption= Movimiento de los Agentes, label=codigo6, language=Python]
            for i in range(a):
                r = agentes.iloc[i]
                
                xl = r.x - r.dx
                xr = r.x + r.dx
                yd = r.y - r.dy
                yu = r.y + r.dy
                
                if  xl < low+step:
                    xl = r.x
                if  xr > high-step:
                    xr = r.x
                if  yd < low+step:
                    yd = r.y
                if  yu > high-step:
                    yu = r.y
                
                g1 = g(xl, yu)
                g2 = g(r.x, yu)
                g3 = g(xr, yu)
                g4 = g(xl, r.y)
                g5 = g(xr, r.y)
                g6 = g(xl, yd)
                g7 = g(r.x, yd)
                g8 = g(xr, yd)
                lista = [g1,g2,g3,g4,g5,g6,g7,g8]
\end{lstlisting}

Por \'ultimo, se decide la direcci\'on en que la funci\'on toma un valor menor y se reescriben las coordenadas de los agentes. Adem\'as, se toman las coordenadas del mayor de los valores evaluados de la funci\'on y se guardan para poder graficar el punto posteriormente, como se aprecia en el c\'odigo \ref{codigo7}.

\begin{lstlisting}[caption= Maximizaci\'on de la Funci\'on, label=codigo7, language=Python]
                mayor = lista.index(max(lista))+1   
                       
                if mayor == 1:
                    agentes.at[i, 'x'] = xl
                    agentes.at[i, 'y'] = yu
                elif mayor ==2:
                    agentes.at[i, 'x'] = r.x
                    agentes.at[i, 'y'] = yu
                elif mayor ==3:
                    agentes.at[i, 'x'] = xr
                    agentes.at[i, 'y'] = yu
                elif mayor ==4:
                    agentes.at[i, 'x'] = xl
                    agentes.at[i, 'y'] = r.y
                elif mayor ==5:
                    agentes.at[i, 'x'] = xr
                    agentes.at[i, 'y'] = r.y
                elif mayor ==6:
                    agentes.at[i, 'x'] = xl
                    agentes.at[i, 'y'] = yd
                elif mayor ==7:
                    agentes.at[i, 'x'] = r.x
                    agentes.at[i, 'y'] = yd
                elif mayor ==8:
                    agentes.at[i, 'x'] = xr
                    agentes.at[i, 'y'] = yd
                
                mejor = g(r.x, r.y)
                if mejor > best:
                    best = g(r.x, r.y)
                    bestx = r.x
                    besty = r.y
                
                if mejor > r.best:
                    agentes.at[i, 'best'] = mejor
\end{lstlisting}

El \href{https://github.com/FeroxDeitas/Simulacion-Nano/blob/main/Tareas/P7/busqueda_local.py}{desarrollo} completo del c\'odigo puede encontrarse en el repositorio de J. Torres en GitHub \cite{jorge1}.

\section{Resultados}\label{res}
En su repositorio de GitHub, J. Torres presenta una \href{https://github.com/FeroxDeitas/Simulacion-Nano/blob/main/Tareas/P7/maximizar.gif}{animaci\'on} tipo GIF del proceso completo que realizan 10 agentes en 1000 pasos para encontrar los valores m\'aximos de la funci\'on evaluada, mientras que en la figura \ref{fig1} se pueden observar algunos de estos pasos. La proyecci\'on plana est\'a codificada por colores, en donde el azul oscuro representa los menores valores y el amarillo brillante representa los mayores valores que toma la funci\'on al ser evaluada en el rango definido. El eje horizontal representa los valores de las $x$, mientras que el eje vertical representa los valores de las $y$. Cada uno de los puntos rojos que se observan es un agente que realiza un movimiento, $\Delta x$ y $\Delta y$, en cada paso de la iteraci\'on, mientras que el punto verde es el m\'aximo valor encontrado en dicho paso. 

\begin{figure}
     \begin{subfigure}[b]{0.3\textwidth}
         \centering
         \includegraphics[width=\textwidth]{images/p7_v_t000.png}
         \caption{\ }
     \end{subfigure}
     \begin{subfigure}[b]{0.3\textwidth}
         \centering
         \includegraphics[width=\textwidth]{images/p7_v_t010.png}
         \caption{\ }
     \end{subfigure}
     \begin{subfigure}[b]{0.3\textwidth}
         \centering
         \includegraphics[width=\textwidth]{images/p7_v_t020.png}
         \caption{\ }
     \end{subfigure}
     \begin{subfigure}[b]{0.3\textwidth}
         \centering
         \includegraphics[width=\textwidth]{images/p7_v_t030.png}
         \caption{\ }
     \end{subfigure}
     \begin{subfigure}[b]{0.3\textwidth}
         \centering
         \includegraphics[width=\textwidth]{images/p7_v_t040.png}
         \caption{\ }
     \end{subfigure}
     \begin{subfigure}[b]{0.3\textwidth}
         \centering
         \includegraphics[width=\textwidth]{images/p7_v_t050.png}
         \caption{\ }
     \end{subfigure}
     \begin{subfigure}[b]{0.3\textwidth}
         \centering
         \includegraphics[width=\textwidth]{images/p7_v_t060.png}
         \caption{\ }
     \end{subfigure}
     \begin{subfigure}[b]{0.3\textwidth}
         \centering
         \includegraphics[width=\textwidth]{images/p7_v_t070.png}
         \caption{\ }
     \end{subfigure}
     \begin{subfigure}[b]{0.3\textwidth}
         \centering
         \includegraphics[width=\textwidth]{images/p7_v_t080.png}
         \caption{\ }
     \end{subfigure}
     \begin{subfigure}[b]{0.3\textwidth}
         \centering
         \includegraphics[width=\textwidth]{images/p7_v_t090.png}
         \caption{\ }
     \end{subfigure}
     \caption{Movimientos de agentes y maximizaci\'on de la funci\'on.}
     \label{fig1}
\end{figure}
\newpage
\section{Conclusiones}\label{con}
Como se puede observar en la figura \ref{fig1}, \'esta t\'ecnica es bastante vers\'atil y logra encontrar los valores m\'aximos (o m\'inimos) de una funci\'on. Sin embargo, presenta algunas dificultades al tratarse de una funci\'on oscilatoria, pues los puntos pueden caer en un ciclo en donde no pueden salir de una cresta o un valle y, por lo tanto, no encuentran un m\'aximo o m\'inimo absoluto.

\bibliography{tarea_7}
\bibliographystyle{plainnat}

\end{document}
